\documentclass[12pt,letterpaper,oneside,reqno]{amsart}
\usepackage{amsfonts}
\usepackage{amsmath}
\usepackage{amssymb}
\usepackage{amsthm}
\usepackage{float}
\usepackage{mathrsfs}
\usepackage{colonequals}
\usepackage[font=small,labelfont=bf]{caption}
\usepackage[left=1in,right=1in,bottom=1in,top=1in]{geometry}
\usepackage[pdfpagelabels,hyperindex,colorlinks=true,linkcolor=blue,urlcolor=magenta,citecolor=green]{hyperref}
\usepackage{graphicx}
\linespread{1.7}
\emergencystretch=1em
\usepackage{array}
\usepackage{etoolbox}
\apptocmd{\sloppy}{\hbadness 10000\relax}{}{}
\raggedbottom

\newcommand \anglePower [2]{\langle #1 \rangle \sp{#2}}
\newcommand \bernoulli [2][B] {{#1}\sb{#2}}
\newcommand \curvePower [2]{\{#1\}\sp{#2}}
\newcommand \coeffA [3][A] {{\mathbf{#1}} \sb{#2,#3}}
\newcommand \polynomialP [4][P]{{\mathbf{#1}}\sp{#2} \sb{#3}(#4)}

% ordinary derivatives
\newcommand \derivative [2] {\frac{d}{d #2} #1}                              % 1 - function; 2 - variable;
\newcommand \pderivative [2] {\frac{\partial #1}{\partial #2}}               % 1 - function; 2 - variable;
\newcommand \qderivative [1] {D_{q} #1}                                      % 1 - function
\newcommand \nqderivative [1] {D_{n,q} #1}                                   % 1 - function
\newcommand \qpowerDerivative [1] {\mathcal{D}_q #1}                         % 1 - function;
\newcommand \finiteDifference [1] {\Delta #1}                                % 1 - function;
\newcommand \pTsDerivative [2] {\frac{\partial #1}{\Delta #2}}               % 1 - function; 2 - variable;

% high order derivatives
\newcommand \derivativeHO [3] {\frac{d^{#3}}{d {#2}^{#3}} #1}                % 1 - function; 2 - variable; 3 - order
\newcommand \pderivativeHO [3]{\frac{\partial^{#3}}{\partial {#2}^{#3}} #1}
\newcommand \qderivativeHO [2] {D_{q}^{#2} #1}                               % 1 - function; 2 - order
\newcommand \qpowerDerivativeHO [2] {\mathcal{D}_{q}^{#2} #1}                % 1 - function; 2 - order
\newcommand \finiteDifferenceHO [2] {\Delta^{#2} #1}                         % 1 - function; 2 - order
\newcommand \pTsDerivativeHO [3] {\frac{\partial^{#3}}{\Delta {#2}^{#3}} #1} % 1 - function; 2 - variable;

% central factorials and related symbols
\newcommand \centralFactorial [2] {#1^{[#2]}}
\newcommand \fallingFactorial [2] {\left(#1 \right)^{\underline{#2}}}
\newcommand{\stirlingii}{\genfrac{\{}{\}}{0pt}{}}
\newcommand{\eulerianNumber}{\genfrac{\langle}{\rangle}{0pt}{}}

% for llceil coeffcient
\newcommand{\nobarfrac}{\genfrac{}{}{0pt}{}}
\def\llceil{\left\lceil\kern-3.5pt\left\lceil}
\def\rrfloor{\right\rfloor\kern-3.5pt\right\rfloor}
\newcommand \llceilCoefficient [3] {\llceil \nobarfrac{#1}{#2} \rrfloor_{#3}}

% rascal numbers etc
\newcommand \rascalNumber [3] {\binom{#1}{#2}_{#3}}
\newcommand \north[0] {\mathbf{North}}
\newcommand \south[0] {\mathbf{South}}
\newcommand \west[0] {\mathbf{West}}
\newcommand \east[0] {\mathbf{East}}

% 1-q pascal notation

\newcommand{\genstirlingI}[3]{%
    \genfrac{[}{]}{0pt}{#1}{#2}{#3}%
}
\newcommand{\genstirlingII}[3]{%
    \genfrac{\{}{\}}{0pt}{#1}{#2}{#3}%
}
\newcommand{\oneQBinomial}[3]{\genstirlingI{}{#1}{#2}^{#3}}

% free foot note
\let\svthefootnote\thefootnote
\newcommand\freefootnote[1]{%
    \let\thefootnote\relax%
    \footnotetext{#1}%
    \let\thefootnote\svthefootnote%
}


\newtheorem{theorem}{Theorem}[section]
\newtheorem{corollary}[theorem]{Corollary}
\newtheorem{lemma}[theorem]{Lemma}
\newtheorem{example}[theorem]{Example}
\newtheorem{conjecture}[theorem]{Conjecture}
\newtheorem{definition}[theorem]{Definition}

\numberwithin{equation}{section}

\title[Row sums conjecture in iterated rascal triangles]
{Row sums conjecture in iterated rascal triangles}
\author[Petro Kolosov]{Petro Kolosov}
\address{Software Developer, DevOps Engineer}
\email{kolosovp94@gmail.com}
\urladdr{https://kolosovpetro.github.io}
\keywords{
    Pascal's triangle,
    Rascal triangle,
    Binomial coefficients,
    Binomial identities,
    Binomial theorem,
    Generalized Rascal triangles,
    Iterated rascal triangles,
    Iterated rascal numbers,
    Number triangle,
    Arithmetic sequence,
    Vandermonde identity,
    Vandermonde convolution
}
\subjclass[2010]{11B25,11B99}
\date{\today}
\hypersetup{
    pdftitle={Row sums conjecture in iterated rascal triangles},
    pdfsubject={
        Pascal's triangle,
        Rascal triangle,
        Binomial coefficients,
        Binomial identities,
        Binomial theorem,
        Generalized Rascal triangles,
        Iterated rascal triangles,
        Iterated rascal numbers,
        Number triangle,
        Arithmetic sequence,
        Vandermonde identity,
        Vandermonde convolution
    },
    pdfauthor={Petro Kolosov},
    pdfkeywords={
        Pascal's triangle,
        Rascal triangle,
        Binomial coefficients,
        Binomial identities,
        Binomial theorem,
        Generalized Rascal triangles,
        Iterated rascal triangles,
        Iterated rascal numbers,
        Number triangle,
        Arithmetic sequence,
        Vandermonde identity,
        Vandermonde convolution
    }
}
\begin{document}
    \begin{abstract}
        In~\cite{gregory2023iterated}, Gregory et al.
provide the following conjecture for row sums of iterated rascal triangles.
For every $i$
\begin{align*}
    \sum_{k=0}^{4i+3} \rascalNumber{4i+3}{k}{i} &= 2^{4i+2}
%        \sum_{k=0}^{4i+3} \rascalNumber{4i+3}{k}{i} &= \sum_{k=0}^{4i+3} \sum_{m=0}^{i} \binom{4i+3-k}{m} \binom{k}{m} = 2^{4i+2}
\end{align*}
where $\rascalNumber{n}{k}{i}$ is an iterated rascal number.

    \end{abstract}

    \maketitle

    \tableofcontents

    \freefootnote{Sources: \url{https://github.com/kolosovpetro/RowSumsConjectureInRascalTriangle}}

    \section{Introduction} \label{sec:introduction}
    In~\cite{gregory2023iterated}, Gregory et al.
provide the following conjecture for row sums of iterated rascal triangles.
\begin{conjecture}
    \label{conjecture:row_sums}
    For every $i$
    \begin{align*}
        \sum_{k=0}^{4i+3} \rascalNumber{4i+3}{k}{i} = 2^{4i+2}
    \end{align*}
\end{conjecture}
where $\rascalNumber{n}{k}{i}$ is an iterated rascal number.
Define the iterated rascal number
\begin{definition}
    Iterated rascal number
    \begin{align*}
        \rascalNumber{n}{k}{i} &= \sum_{m=0}^{i} \binom{n-k}{m} \binom{k}{m}
    \end{align*}
\end{definition}
Note that iterated rascal numbers are closely related to Vandermonde convolution
$\binom{a+b}{r} = \sum_{m=0}^{r} \binom{a}{m} \binom{b}{r-m}$
\begin{equation*}
    \rascalNumber{n}{k}{i} = \sum_{m=0}^{i} \binom{n-k}{m} \binom{k}{k-m}
\end{equation*}
While
\begin{equation*}
    \binom{n}{k}= \sum_{m=0}^{k} \binom{n-k}{m} \binom{k}{k-m}
\end{equation*}
It is straightforward to see that
\begin{equation*}
    \binom{n}{k} - \rascalNumber{n}{k}{i} = \sum_{m=i+1}^{k} \binom{n-k}{m} \binom{k}{k-m}
\end{equation*}
In particular, above sum is zero for $k \leq i$, that means
\begin{equation*}
    \binom{n}{k} = \rascalNumber{n}{k}{i}, \quad  0 \leq k \leq i
\end{equation*}
To prove the conjecture~\eqref{conjecture:row_sums}
we utilize above relations in terms of binomial coefficients and iterated rascal numbers.
Recall the row sums property of binomial coefficients
\begin{equation*}
    \sum_{k=0}^{4i+3} \binom{4i+3}{k} = 2^{4i+3}
\end{equation*}
If conjecture~\eqref{conjecture:row_sums} is true, then it is also true that
\begin{equation*}
     \sum_{k=0}^{4i+3} \binom{4i+3}{k} -  \sum_{k=0}^{4i+3} \binom{4i+3}{k}_i = 2^{4i+2}
\end{equation*}
because $2^{4i+3} - 2^{4i+2} = 2^{4i+2}$.
Expanding both sums we get
\begin{align*}
    2^{4i+3} &= \sum_{k=0}^{4i+3} \sum_{m=0}^{k} \binom{4i+3-k}{m} \binom{k}{k-m} - \sum_{k=0}^{4i+3} \sum_{m=0}^{i} \binom{4i+3-k}{m} \binom{k}{k-m}\\
    2^{4i+3} &= \sum_{k=0}^{4i+3} \sum_{m=0}^{k} \binom{4i+3-k}{m} \binom{k}{k-m} - \sum_{m=0}^{i} \binom{4i+3-k}{m} \binom{k}{k-m}
\end{align*}
Note that $\binom{n}{k} \geq \binom{n}{k}_i$ for each $n,k,i$.



    \section{Conclusions}\label{sec:conclusions}
    Conclusions of your manuscript.

    \bibliographystyle{unsrt}
    \bibliography{RowSumsConjectureInRascalTriangle}
    \noindent \textbf{Version:} \texttt{Local-0.1.0}

\end{document}
