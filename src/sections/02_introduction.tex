Rascal triangle is Pascal-like numeric triangle developed in 2010 by three middle school students,
Alif Anggoro, Eddy Liu, and Angus Tulloch~\cite{anggoro2010rascal}.
During math classes they were challenged to provide the next row for the following number triangle
\[
    \begin{array}{cccccccc}
        &   &   &   & 1 &   &   &   \\
        &   &   & 1 &   & 1 &   &   \\
        &   & 1 &   & 2 &   & 1 &   \\
        & 1 &   & 3 &   & 3 &   & 1 \\
        & & & & \dots & &
    \end{array}
\]

Teacher's expected answer was the one that matches Pascal's triangle, e.g ``1 4 6 4 1''.
However, Anggoro, Liu, and Tulloch suggested ``1 4 5 4 1'' instead.
They devised this new row via what they called diamond formula
\begin{align*}
    \south  = \frac{\east \cdot \west + 1}{\north}
\end{align*}
So they obtained the following triangle
\begin{table}[H]
    \begin{center}
        \setlength\extrarowheight{-6pt}
        \begin{tabular}{c|cccccccc}
            $n/k$ & 0 & 1 & 2  & 3  & 4  & 5  & 6 & 7 \\
            \hline
            0     & 1 &   &    &    &    &    &   &   \\
            1     & 1 & 1 &    &    &    &    &   &   \\
            2     & 1 & 2 & 1  &    &    &    &   &   \\
            3     & 1 & 3 & 3  & 1  &    &    &   &   \\
            4     & 1 & 4 & 5  & 4  & 1  &    &   &   \\
            5     & 1 & 5 & 7  & 7  & 5  & 1  &   &   \\
            6     & 1 & 6 & 9  & 10 & 9  & 6  & 1 &   \\
            7     & 1 & 7 & 11 & 13 & 13 & 11 & 7 & 1
        \end{tabular}
    \end{center}
    \caption{Rascal triangle.}
    \label{tab:rascal-triangle-i-1}
\end{table}
Indeed, the forth row is ``1 4 5 4 1'' because $4= \frac{1 \cdot 3 + 1}{1}$ and $5 = \frac{3 \cdot 3 + 1}{2}$.
Since then, a lot of work has been done over the topic of rascal triangles.
In this article we stick our attention to the one of rascal triangles generalizations,
namely iterated rascal triangles~\cite{gregory2023iterated}.
Iterated rascal number is defined via a sum of binomial coefficients multiplication
\begin{definition}
    Iterated rascal number
    \begin{align}
        \rascalNumber{n}{k}{i} = \sum_{m=0}^{i} \binom{n-k}{m} \binom{k}{m}
    \end{align}
\end{definition}
Thus, the rascal triangle~\eqref{tab:rascal-triangle-i-1} is the triangle generated by $\rascalNumber{n}{k}{1}$.
