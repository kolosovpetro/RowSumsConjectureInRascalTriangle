Let $n=i$.
We need to find the following sum:
\begin{align*}
\sum\limits_{m=0}^n \sum\limits_k\binom{4n+3-k}{m}\binom{k}{m} \quad (1.0)
\end{align*}
Note that we can assume that $k$ goes over all integers without bounds.
We can rewrite the inner sum as
\begin{align*}
[x^{4n+3}]
    \sum\limits_{i,j} \binom{i}{m} x^i \binom{j}{m} x^j &= [x^{4n+3}] \frac{x^m}{(1-x)^{m+1}} \frac{x^m}{(1-x)^{m+1}} \quad (2.0) \\
    &= [x^{4n+3}] \frac{x^{2m}}{(1-x)^{2m+2}}
\end{align*}
which evaluates to
\begin{align*}
    \binom{4n+4}{2m+1} \quad (3.0)
\end{align*}
meaning that the sum is same as $S=\sum\limits_{m=0}^n \binom{4n+4}{2m+1}$.
Note that
\begin{align*}
    \binom{4n+4}{2m+1} = \binom{4n+4}{4n+4-(2m+1)} = \binom{4n+4}{2(2n+1-m)+1}
\end{align*}
Therefore, by substitution $m \to 2n+1-m$, we have
\begin{align*}
    S =\sum\limits_{m=0}^n \binom{4n+4}{2(2n+1-m)+1} = \sum\limits_{m=n+1}^{2n+1} \binom{4n+4}{2m+1}
\end{align*}
But this means that $2S = \sum\limits_{m=0}^{2n+1} \binom{4n+4}{2m+1} =2^{4n+3}$, which implies $S = 2^{4n+2}$. $\square$

P.S. Note that, more generally, it means that $\sum\limits_k  \binom{t-k}{m} \binom{k}{m} = \binom{t+1}{2m+1}$, so we have
\begin{align*}
    \sum\limits_{k=0}^t \binom{t}{k}_n = \sum\limits_{m=0}^n \binom{t+1}{2m+1}
\end{align*}
P.S.S.\ Also see [this question](https://math.stackexchange.com/questions/73015) for the identity above.
